\documentclass[aps,prd,onecolumn,superscriptaddress,nofootinbib]{revtex4-2}

% — Packages —
\usepackage[utf8]{inputenc}
\usepackage[T1]{fontenc}
\usepackage{lmodern}
\usepackage{amsmath,amssymb,bm,mathtools,amsthm}
\usepackage{graphicx}
\usepackage{adjustbox}
\usepackage{xcolor}
\usepackage{microtype}
\usepackage[unicode, pdfencoding=auto, psdextra]{hyperref}
\usepackage{enumitem}
\usepackage{tikz}
\usetikzlibrary{arrows.meta,positioning,fit,calc,shapes.geometric,shapes.multipart,backgrounds}

% — Colors for flowchart —
\definecolor{flowBlue}{HTML}{1F77B4}
\definecolor{flowPurple}{HTML}{9467BD}
\definecolor{flowGreen}{HTML}{2CA02C}
\definecolor{flowOrange}{HTML}{FF7F0E}
\definecolor{flowRed}{HTML}{D62728}

% — Hyperref —
\hypersetup{
  colorlinks=true,
  linkcolor=blue,
  citecolor=blue,
  urlcolor=blue
}

% — PDF string sanitization —
\pdfstringdefDisableCommands{%
\def\OmL{OmegaLambda}%
\def\Omm{Omega m0}%
\def\cgeo{cgeo}%
\def\alphaM{alphaM}%
\def\eps{epsilon}%
\def\boxed#1{#1}%
\def\mu{mu}%
\def\alpha{alpha}%
\def\alpha_M{alphaM}%
\def\Omega_\Lambda{OmegaLambda}%
}

% — Math tweaks —
\allowdisplaybreaks

% — Macros —
\providecommand{\mpl}{M_{\rm P}}
\providecommand{\OmL}{\Omega_\Lambda}
\providecommand{\Omm}{\Omega_{m0}}
\providecommand{\cgeo}{c_{\rm geo}}
\providecommand{\alphaM}{\alpha_M}
\providecommand{\eps}{\varepsilon}

% — Theorem-like environments —
\newtheorem{definition}{Definition}
\newtheorem{hypothesis}{Hypothesis}
\newtheorem{lemma}{Lemma}
\newtheorem{proposition}{Proposition}
\newtheorem{theorem}{Theorem}

\begin{document}

\title{State-Dependent Gravity from Modular Information:\\
A KMS/FDT Linear-Response Framework (Conditional)}

\author{[Authors]}
\affiliation{[Institutions]}
\date{}

\begin{abstract}
We present a \emph{conditional} information-theoretic framework in which finite local information capacity produces a \emph{state-dependent} gravitational response. The key working assumption replaces macroscopic Clausius language with a \textbf{KMS-normalized linear-response (A2--KMS) hypothesis}: in the small-wedge, MI/moment-kill projector channel, Bisognano--Wichmann (BW) KMS structure and the fluctuation--dissipation theorem (FDT) fix the sign and normalization of the modular susceptibility to \(\mathcal O(\ell^4)\), with curvature/contact remainders \(\mathcal O(\ell^6)\). We \emph{define} a dimensionless state variable \(\eps\) via flat-space modular response, \(\delta\!\langle K_{\rm sub}\rangle=(2\pi C_T I_{00})\,\ell^4\,\delta\eps+\mathcal O(\ell^6)\), and \emph{map} \(\eps\) to a weak-field coupling through A2--KMS, yielding \(\delta G/G=-\beta\,\delta\eps\) with \(\beta\equiv 2\pi C_T I_{00}\). A geometric normalization yields a universal weak-field prefactor \(5/12=(4/3)\times(5/16)\), implying \(\mu(\eps)=1/(1+\tfrac{5}{12}\eps)\) and \(a_0=\tfrac{5}{12}\,\OmL^2 cH_0\). The scheme-invariant mapping \(\OmL=\beta\,f\,\cgeo\approx 0.685\) (conservative \(\pm 5\%\) from shared systematics in \(\beta\)) preserves EM/GW distances (distance sector kept GR-like). We formalize the \emph{entropy-driven} history \(\eps(a)\) by a retarded KMS susceptibility: a positive integrated kernel enforces \(d\eps/d\ln a\ge 0\) (FDT-positivity), with normalization fixed by \(\int \eps\,d\ln a=\OmL\). Substrate \emph{structural consistency checks} (HQTFIM and Gaussian chains) confirm algebraic ingredients (first-law channel, constant+log size trend, MI/moment-kill plateau, FDT-positivity) but are \emph{not} 4D curved-spacetime surrogates. We give explicit falsifiers, conservative uncertainties, and a limitations box (safe-window viability, CHM-vs-half-space KMS error \(\sim \mathcal O((\ell/L_{\rm curv})^2)\), environment gate micro\-physics, quantum-classical bridge). This is an exploratory, testable, and conditional framework rather than a phenomenological fit.
\end{abstract}

\maketitle

% ===============================
\section{Scope, Working Order, and Limitations (Read First)}
\label{sec:scope}
\noindent\textbf{Working order.} Throughout, ``working order'' means we isolate the isotropic \(\ell^4\) contribution in the MI/moment-kill projector channel and treat curvature/contact corrections as \(\mathcal O(\ell^6)\).\\[3pt]
\noindent\textbf{Safe window (existence is model dependent).} We assume a nonempty range \(\ell\) obeying
\[
\epsilon_{\rm UV}\ll \ell \ll \min\{L_{\rm curv},\lambda_{\rm mfp},m_i^{-1}\}.
\]
In halos with \(L_{\rm curv}\!\sim\!10\)~Mpc, a plausible late-time band is \(\ell\in[1,100]\)~pc; this window can be \emph{absent} in dense regions (star-forming zones, cluster cores).\\[3pt]
\noindent\textbf{KMS applicability (CHM vs.\ half-space).} Exact BW KMS analyticity holds for half-spaces; CHM diamonds approximate it in the safe window. The fractional KMS deviation scales as \(\mathcal O((\ell/L_{\rm curv})^2)\) (App.~\ref{app:chm-kms-estimate}).\\[3pt]
\noindent\textbf{Distances kept GR-like.} We enforce \(\alphaM\simeq 0\) in the distance sector; null geometry and EM/GW distances are unmodified at working order.\\[3pt]
\noindent\textbf{Environment gate is illustrative.} The gate \(F_g(g/a_0)\) is a minimal compliance envelope: \(F_g\!\to\!0\) in strong fields (Solar System), \(F_g\!\to\!1\) in weak fields; a microscopic derivation is future work.\\[3pt]
\noindent\textbf{Substrate tests are algebraic checks.} HQTFIM/Gaussian runs test the algebraic structure (first-law channel, constant+log trend, plateau, FDT-positivity). They are \emph{not} physical surrogates for 4D curved spacetime.\\[3pt]
\noindent\textbf{Falsifiers and uncertainties.} We list sharp falsifiers (Sec.~\ref{sec:falsifiers}) and adopt a conservative \(\pm 5\%\) uncertainty on \(\beta\) (shared systematics), with angle-invariance presented as a \emph{null} residual test rather than a precision claim.

% ===============================
\section{A2--KMS Hypothesis (Definition, Channel, and Positivity)}
\label{sec:A2KMS}
\paragraph{BW recap.} The Minkowski vacuum restricted to a Rindler half-space is a KMS state at inverse temperature \(\beta_{\rm KMS}=2\pi/\kappa\) with respect to boost flow (Bisognano--Wichmann).\\[3pt]
\noindent\textbf{A2--KMS (boxed).}
\begin{hypothesis}[A2--KMS (working order)]
\label{hyp:A2KMS}
In the MI/moment-kill projector channel for small CHM diamonds in a safe window, the wedge state inherits BW KMS analyticity up to \(\mathcal O((\ell/L_{\rm curv})^2)\). The linear-response susceptibility relating modular perturbations to boost-energy flux is fixed by the KMS two-point function, is positive (FDT), and its finite \(\ell^4\) coefficient equals the flat-space value at working order:
\[
\delta\!\langle K_{\rm sub}\rangle=(2\pi C_T I_{00})\,\ell^4\,\delta\eps+\mathcal O(\ell^6),\quad
\frac{\delta G}{G}=-\beta\,\delta\eps,\;\;\beta\equiv 2\pi C_T I_{00}.
\]
\end{hypothesis}
\noindent\emph{Remarks.} (i) Exact KMS is half-space; diamond validity is approximate and quantified in App.~\ref{app:chm-kms-estimate}. (ii) FDT positivity enforces \(\Delta S\ge 0\) in this channel without invoking macroscopic heat. (iii) ``Temperature'' is the KMS normalization for boost flow, not a literal bath.

% ===============================
\section{Definition vs.\ Mapping (Separation of Roles)}
\label{sec:def-vs-map}
\paragraph{Definition (flat-space QFT).} We \emph{define} \(\eps(x)\) by the MI-subtracted modular response in flat space:
\begin{equation}
\label{eq:eps-def}
\delta\!\langle K_{\rm sub}(\ell)\rangle = \underbrace{(2\pi C_T I_{00})}_{\beta}\,\ell^4\,\delta\eps(x) + \mathcal O(\ell^6).
\end{equation}
\paragraph{Mapping (A2--KMS).} We \emph{map} \(\eps\) to a response via A2--KMS:
\begin{equation}
\label{eq:mapping}
\frac{\delta G}{G}=-\beta\,\delta\eps,\qquad \beta=2\pi C_T I_{00}.
\end{equation}
The roles are distinct; no circularity arises.

% ===============================
% — Vertical, color-coded pipeline figure (stable links, no fragile hyperref in nodes) —
\begin{figure}[t]
\centering
\begin{adjustbox}{max width=\linewidth}
\begin{tikzpicture}[
  node distance=3.2mm and 10mm,
  every node/.style={font=\footnotesize},
  stageH/.style={draw, rounded corners=2pt, align=center, inner sep=4pt, outer sep=0pt, text width=0.78\linewidth, font=\footnotesize\bfseries},
  stage/.style={draw, rounded corners=2pt, align=center, inner sep=4pt, outer sep=0pt, text width=0.78\linewidth},
  spur/.style={draw, rounded corners=2pt, align=left, inner sep=4pt, outer sep=0pt, text width=0.78\linewidth},
  arr/.style={-Latex, semithick, shorten >=2pt, shorten <=2pt},
  darr/.style={-Latex, dashed, semithick, shorten >=2pt, shorten <=2pt}
]
% Blue header and steps
\node[stageH, draw=flowBlue!80, fill=flowBlue!20] (B0) {[Blue: QFT / Modular Analysis]};
\node[stage, below=of B0, draw=flowBlue!80, fill=flowBlue!8] (B1) {Flat-space QFT $\to$ MI-subtracted modular Hamiltonian (CHM/OP) (Sec.~\ref{sec:beta})};
\node[stage, below=of B1, draw=flowBlue!80, fill=flowBlue!8] (B2) {Compute $\beta = 2\pi\,C_T\,I_{00}$ (four routes; $\pm 5\%$ shared; Sec.~\ref{sec:beta})};
\draw[arr] (B0) -- (B1); \draw[arr] (B1) -- (B2);

% validation spurs
\node[spur, below=of B2, draw=flowBlue!80, fill=flowBlue!4] (S1) {\textbf{Validation Spur 1: Four independent QFT runs}\\• Real-space CHM + MI\\• Spectral/Bessel (momentum)\\• Euclidean time-slicing\\• Replica finite-difference\\(spread $\lesssim 1\%$; adopt $\pm 5\%$ shared)};
\draw[darr] (B2) -- (S1);
\node[spur, below=of S1, draw=flowBlue!80, fill=flowBlue!4] (S2) {\textbf{Spur 2: Substrate structural checks}\\• HQTFIM (linear-window first law; constant+log; plateau$\approx 0$; FDT positivity)\\• Gaussian chain (exact first-law; positivity)};
\draw[darr] (S1) -- (S2);

% Purple mapping
\node[stageH, below=8mm of S2, draw=flowPurple!80, fill=flowPurple!20] (P0) {[Purple: Geometric Mapping]};
\node[stage, below=of P0, draw=flowPurple!80, fill=flowPurple!8] (P1) {Scheme-invariant product $\beta f c_{\rm geo}$ (Sec.~\ref{sec:geom-map})};
\node[stage, below=of P1, draw=flowPurple!80, fill=flowPurple!8] (P3) {$\Omega_\Lambda = \beta f c_{\rm geo} \ \approx\ 0.685$};
\draw[arr] (P0) -- (P1); \draw[arr] (P1) -- (P3);

% Green weak-field
\node[stageH, below=8mm of P3, draw=flowGreen!80, fill=flowGreen!20] (G0) {[Green: Weak-field Sector]};
\node[stage, below=of G0, draw=flowGreen!80, fill=flowGreen!8] (G1) {Universal prefactor $5/12=(4/3)\times(5/16)$ (Sec.~\ref{sec:weakfield})};
\node[stage, below=of G1, draw=flowGreen!80, fill=flowGreen!8] (G2) {$a_0 = (5/12)\,\Omega_\Lambda^2\,c\,H_0$};
\node[stage, below=of G2, draw=flowGreen!80, fill=flowGreen!8] (G4) {$\mu(\varepsilon)=1/(1+\tfrac{5}{12}\varepsilon)$ (growth)};
\draw[arr] (G0) -- (G1); \draw[arr] (G1) -- (G2); \draw[arr] (G2) -- (G4);

% Green2 epsilon(a)
\node[stageH, below=8mm of G4, draw=flowGreen!80, fill=flowGreen!20] (E0) {[Green: Entropy-driven $\varepsilon(a)$ (KMS/FDT)]};
\node[stage, below=of E0, draw=flowGreen!80, fill=flowGreen!8] (E1) {Retarded positive kernel $\Rightarrow d\varepsilon/d\ln a \ge 0$ (Sec.~\ref{sec:epsilon})};
\node[stage, below=of E1, draw=flowGreen!80, fill=flowGreen!8] (E2) {Normalization: $\int \varepsilon\, d\ln a = \Omega_\Lambda$};
\draw[arr] (E0) -- (E1); \draw[arr] (E1) -- (E2);

% Orange observations
\node[stageH, below=8mm of E2, draw=flowOrange!90, fill=flowOrange!25] (O0) {[Orange: Observational Application]};
\node[stage, below=of O0, draw=flowOrange!90, fill=flowOrange!12] (O1) {Growth: $S_8 \approx 0.788$ ($\sim\!-7\%$) (Sec.~\ref{sec:obs})};
\node[stage, below=of O1, draw=flowOrange!90, fill=flowOrange!12] (O2) {Hubble ladder (bounds): $H_0=71.18$ (uncapped SN), $70.89$ (capped SN+Cepheid)};
\node[stage, below=of O2, draw=flowOrange!90, fill=flowOrange!12] (O3) {Distances remain GR-like ($\alpha_M\simeq 0$)};
\draw[arr] (O0) -- (O1); \draw[arr] (O1) -- (O2); \draw[arr] (O2) -- (O3);
\end{tikzpicture}
\end{adjustbox}
\caption{Vertical, color-coded pipeline. Blue: modular QFT leading to $\beta$ (validation spurs underneath). Purple: scheme-invariant mapping $\Omega_\Lambda=\beta f c_{\rm geo}$. Green: weak-field sector ($5/12$, $a_0$, $\mu(\eps)$) and KMS/FDT-driven $\eps(a)$ (monotone; normalization $\int\eps\,d\ln a=\Omega_\Lambda$). Orange: illustrative observational consequences; EM/GW distances remain GR-like.}
\label{fig:pipeline}
\end{figure}

% ===============================
\section{QFT Input: \texorpdfstring{$\beta=2\pi C_T I_{00}$}{beta}}
\label{sec:beta}
We evaluate \(\beta\) via four independent routes sharing only OP/CHM conventions and the MI+moment-kill projector: (a) real-space CHM kernel; (b) spectral/Bessel (momentum-space); (c) Euclidean time-slicing; (d) replica finite-difference. Angle invariance is presented as a \emph{null} residual test (identity by construction). Conservatively,
\begin{equation}
\beta = 0.02086 \pm 0.00105 \quad (5\%~\text{shared systematics}).
\end{equation}
\noindent\textbf{Scheme/angle invariance.} Physical predictions use \(\mathcal C_\Omega\equiv f(\theta)\, \cgeo(\theta)\), which is analytically angle-invariant; we show residuals as a null check rather than a precision measurement (Sec.~\ref{sec:theta}).

% ===============================
\section{Geometric Normalization and Background Mapping}
\label{sec:geom-map}
With the continuous-angle normalization (Sec.~\ref{sec:theta}) the FRW zero mode satisfies the \emph{scheme-invariant} mapping
\begin{equation}
\boxed{\ \OmL = \beta\, f\, \cgeo\ } \qquad \Rightarrow \qquad \OmL \approx 0.685 \ \pm\ 0.034\ \ (\text{from }\pm 5\%~\beta).
\end{equation}
Distances are kept GR-like (\(\alphaM\simeq 0\) in the distance sector); lensing is unaltered at working order.

% ===============================
\section{Weak-Field Sector: \texorpdfstring{$5/12$}{5/12}, \texorpdfstring{$\mu(\eps)$}{mu} and \texorpdfstring{$a_0$}{a0}}
\label{sec:weakfield}
Coarse-graining the KMS susceptibility over the wedge family yields a universal geometric factor \(5/12=(4/3)\times(5/16)\) (App.~\ref{app:five-twelve}). The weak-field response and static normalization read
\begin{equation}
\label{eq:mu-a0}
\mu(\eps)=\frac{1}{1+\tfrac{5}{12}\,\eps},\qquad
a_0=\frac{5}{12}\,\OmL^2\,c\,H_0,
\end{equation}
with the same bookkeeping that fixes the FRW zero mode. The factor \(4/3\) is the isotropic null contraction in the BW channel; its universality follows from the UV (\(w=1/3\)) sector governing the susceptibility (App.~\ref{app:five-twelve}).

% ===============================
\section{Entropy-Driven Evolution of \texorpdfstring{$\varepsilon(a)$}{epsilon(a)}}
\label{sec:epsilon}
\paragraph{KMS/FDT differential constraint (positivity).}
Let \(\hat Q\) denote the boost-energy flux operator in the CHM diamond and \(\chi_{QK}\) the retarded susceptibility between \(\hat Q\) and the MI-subtracted modular generator \(\hat K_{\rm sub}\). In linear response,
\[
\delta\!\langle \hat Q\rangle(a) = \int^{\ln a}\! d\ln a' \; \chi_{QK}(a,a')\, \delta\!\langle \hat K_{\rm sub}\rangle(a'),
\]
and FDT with KMS normalization implies \(\int \chi_{QK}\,d\ln a'\ge 0\) in the projector channel. Parameterizing the (dimensionless) throughput intensity by a nonnegative functional \(\mathcal I(a)\), we write the \textbf{entropy-driven law}
\begin{equation}
\label{eq:eps-ode}
\boxed{\;\frac{d\eps}{d\ln a} = \sigma(a)\,\mathcal I(a)\quad\text{with}\quad \sigma(a)\ge 0,\ \ \mathcal I(a)\ge 0\;}
\end{equation}
so that \(\Delta S\ge 0\Rightarrow d\eps/d\ln a\ge 0\) (monotone). This KMS/FDT constraint is a \emph{physical principle}, not a retrofitted choice.

\paragraph{Normalization by the background mapping.}
The scheme-invariant background relation fixes the total ``budget''
\begin{equation}
\label{eq:budget}
\boxed{\;\int_{a_{\rm i}}^{1}\!\eps(a)\, d\ln a \;=\; \OmL \;=\; \beta\, f\,\cgeo\; ,\;}
\end{equation}
so once \(\mathcal I(a)\) and \(\sigma(a)\) are specified by microphysics, \(\eps(a)\) is determined up to an initial condition \(\eps(a_{\rm i})\equiv \eps_0\ge 0\) (irreversibility floor).

\paragraph{A minimal illustrative family (used in Sec.~\ref{sec:obs}).}
As a concrete but non-unique realization consistent with Eq.~\eqref{eq:eps-ode}, define an exposure
\begin{equation}
\label{eq:Jdef}
J(a)=\int^{\ln a}\! d\ln a'\; K(a,a')\, \Phi(a'),\qquad K(a,a')\propto (a'/a)^p,\ \ p\in[4,6],\ \ \Phi\ge 0,
\end{equation}
and set
\begin{equation}
\label{eq:eps-form}
\eps(a)=\eps_0+c_{\log}\,\ln\!\Big(1+\frac{J(a)}{J_*}\Big),\qquad \frac{d\eps}{d\ln a}=\frac{c_{\log}}{1+J/J_*}\,\frac{dJ}{d\ln a}\ \ge 0.
\end{equation}
The normalization constant \(c_{\log}\) is fixed by Eq.~\eqref{eq:budget}. This family enforces monotonicity and the budget while leaving \(\eps_0\) and the kernel details to microphysics.

\paragraph{What is fixed vs.\ what remains free.}
\emph{Fixed by physics:} (i) monotonicity \(d\eps/d\ln a\ge 0\) (KMS/FDT); (ii) total budget \(\int \eps\, d\ln a=\OmL\) (background mapping); (iii) strong-field recovery via environment gating in observables. \emph{Remaining freedom:} (i) initial floor \(\eps_0\ge 0\); (ii) the precise retarded kernel \(K(a,a')\) and driver \(\Phi(a')\) (we bracket with \(p\in[4,6]\)); (iii) a scale \(J_*\). In practice, our headline growth number \(S_8\simeq 0.788\) is \emph{insensitive} to \(p\) within \([4,6]\) at the \(<10^{-3}\) level, indicating limited tuning. The Hubble-ladder bounds are likewise presented as \emph{bounds}, not fits.

% ===============================
\section{Structural Consistency Checks (Substrates)}
\label{sec:substrates}
We implement two independent microscopic testbeds to check the \emph{algebraic} ingredients: (i) an interacting HQTFIM chain (exact diagonalization); (ii) a Gaussian (free-fermion) chain via correlation matrices. These confirm: (1) first-law channel in the linear window; (2) constant+log dependence of \(\delta\!\langle K\rangle(\ell)\); (3) near-zero plateau after subtracting \([1,\log \ell]\); (4) \textbf{FDT positivity} in the projected channel (integrated susceptibility nonnegative; exact for Gaussian, numerically for HQTFIM within tolerance). These are \emph{not} curved 4D surrogates.

% ===============================
\section{Observational Consequences (Illustrative Bounds)}
\label{sec:obs}
\paragraph{Growth.} With \(\mu(\eps)\) of Eq.~\eqref{eq:mu-a0} and an entropy-driven \(\eps(a)\) consistent with Eqs.~\eqref{eq:eps-ode}–\eqref{eq:eps-form}, we find \(S_8\simeq 0.788\) (about \(-7\%\) vs \(\Lambda\)CDM), robust to kernel powers \(p\in[4,6]\) at the \(<10^{-3}\) level.\\
\paragraph{Hubble ladder (capped illustration).} Using an environment gate \(F_g(g/a_0)\) as a minimal compliance envelope (Solar-System recovery, weak-field throttling), an SH0ES-like catalog shifts \(H_0\!:\,73.0\to 71.18\) (uncapped SN) and to \(70.89\) (capped SN+Cepheid). These are \emph{bounds}, not fits; distances remain GR-like.

% ===============================
\section{Relation to EFT-of-DE (Horndeski) and \texorpdfstring{$f(R)$}{f(R)}}
\label{sec:eft}
Linearized about FRW, our closure lives in the \(c_T=1\), no-braiding corner (\(\alpha_T=\alpha_B=0\)) with a single background function \(\alphaM(a)=d\ln M^2/d\ln a\) \cite{BelliniSawicki2014}. Distances are kept GR-like by setting \(\alphaM\simeq 0\) in the distance sector, while the growth sector is modified by the scale-independent \(\mu(\eps)=1/(1+\tfrac{5}{12}\eps)\). In the quasi-static language this corresponds to \(\mu(a)\neq 1\) and \(\Sigma(a)\simeq 1\) (lensing unaltered). By contrast, typical \(f(R)\) models induce scale-dependent \(\mu(k,a)\) and nonzero slip; our mapping is scale-independent at working order and enforces GR-like lensing by construction.

% ===============================
\section{Falsifiers and Honest Gaps}
\label{sec:falsifiers}
\textbf{Falsifiers.} (i) Persistent \(\ell^4\log\ell\) residuals in the MI/moment-kill projector channel; (ii) GW/EM luminosity-distance ratio violating \(|d_L^{\rm GW}/d_L^{\rm EM}-1|\le 5\times 10^{-3}\); (iii) laboratory/solar-system bounds implying \(|\dot G/G|\gtrsim 10^{-12}\,{\rm yr}^{-1}\); (iv) precision cosmology yielding \(\OmL\) inconsistent with \(\beta\,f\,\cgeo\).\\
\textbf{Honest gaps.} (a) Microscopic derivation of the environment gate \(F_g\); (b) quantum-to-classical bridge from modular perturbations to Mpc-scale GR perturbations (likely via coarse-grained RG, entanglement hydrodynamics, noise kernels); (c) rigorous KMS deviation bounds for CHM diamonds (would require full Hadamard parametrix construction).

% ===============================
\section{Angle Invariance (Null-Residual Test)}
\label{sec:theta}
We use a continuous-angle normalization with a unit--solid--angle boundary factor and a cap \(\Delta\Omega(\theta)\). The product \(\mathcal C_\Omega\equiv f(\theta)\, \cgeo(\theta)\) is analytically \(\theta\)-independent; numerically we treat residuals as a \emph{null} check rather than a precision measurement, since the conservative \(\pm 5\%\) \(\beta\) uncertainty dominates.

% ===============================
\section{Data and Code Availability}
\label{sec:data}
Two single-file runners reproduce the substrate checks: \texttt{hqtfim\_capacity\_probe.py} and \texttt{gaussian\_capacity\_probe.py}. They have no cosmological inputs and are intended to validate structural ingredients (first-law channel, constant+log trend, plateau, FDT-positivity).

% ===============================
\section{Conclusion}
We have reframed the core working assumption as a KMS-normalized linear-response hypothesis (A2--KMS), eliminating macroscopic Clausius language while preserving quantitative results. Within a safe window and to working order, the MI/moment-kill projector isolates a finite \(\ell^4\) modular coefficient (flat-space value), FDT positivity enforces \(\Delta S\ge 0\) and thus \(d\eps/d\ln a\ge 0\), and a universal \(5/12\) factor fixes both the weak-field response and the static acceleration scale. The scheme-invariant mapping \(\OmL=\beta f \cgeo\) (with conservative \(\pm 5\%\) on \(\beta\)) maintains GR-like distances and yields sharp falsifiers. This remains a \emph{conditional, exploratory} framework with honest limitations and clear paths for future work.

% ===============================
\appendix

\section{MI subtraction and moment-kill}
\label{app:MI}
Choose coefficients such that, for any smooth radial \(F(r)=F_0+F_2 r^2+\cdots\),
\[
\int_{B_\ell}W_\ell F - a\!\int_{B_{\sigma_1\ell}}W_{\sigma_1\ell}F - b\!\int_{B_{\sigma_2\ell}}W_{\sigma_2\ell}F
=\mathcal O(\ell^6),
\]
canceling \(r^0\) and \(r^2\) moments. The surviving \(\ell^4\) piece defines \(I_{00}\).

\section{Numerical details and uncertainty budget for \texorpdfstring{$\beta$}{beta}}
\label{app:beta}
Four routes (real-space CHM, spectral/Bessel, Euclidean slicings, replica finite-difference) agree within \(\lesssim 1\%\) when scanned over MI windows, gaps, and grids. We adopt a conservative \(\pm 5\%\) overall to account for shared systematics (discretization/regularization). Angle invariance is an identity; we present residuals as a null check rather than a precision claim.

\section{Continuous-angle normalization (invariance identity)}
\label{app:angle}
With a unit--solid--angle boundary factor and \(\Delta\Omega(\theta)=2\pi(1-\cos\theta)\), define \(\cgeo(\theta)=4\pi/\Delta\Omega(\theta)\). The product \(f(\theta)\,\cgeo(\theta)\) becomes independent of \(\theta\) after enforcing no-double-counting of the wedge family; we use this analytically as an invariance identity.

\section{Weak-field flux normalization and the universal \texorpdfstring{$5/12$}{5/12}}
\label{app:five-twelve}
\subsection*{Isotropic null contraction \(4/3\) (BW channel)}
Work in the local rest frame \(u^a\) with spatial projector \(h^{ab}=g^{ab}+u^a u^b\). For future--directed nulls \(k^a\) normalized by \(k^0=|\mathbf{k}|\), angular averaging gives
\[
\big\langle k^a k^b\big\rangle_{\mathbb{S}^2}=(k^0)^2\!\left(u^a u^b+\tfrac{1}{3}h^{ab}\right),\quad
\langle k^0 k^i\rangle=0,\quad \langle k^i k^j\rangle=\tfrac{1}{3}(k^0)^2\delta^{ij}.
\]
For an isotropic stress \(T_{ab}=\rho\,u_a u_b + p\,h_{ab}\), the BW isotropic channel yields
\[
\big\langle T_{ab}k^a k^b\big\rangle_{\mathbb{S}^2}=(k^0)^2(\rho+p)=(k^0)^2(1+w)\rho.
\]
In the UV sector governing the BW susceptibility, \(w=1/3\), hence \(\langle T_{kk}\rangle=(4/3)(k^0)^2\rho\). \textit{This factor is universal for the high-energy sector} (independent of IR modifications).

\subsection*{Geometric segment ratio \(5/16\)}
Averaging the generator density over the CHM wedge family yields the dimensionless segment ratio
\[
R_{\rm seg}=\frac{\int_0^1 u(1-u^2)\hat\rho(u)\,du}{\int_0^1 (1-u^2)\hat\rho(u)\,du}=\frac{5}{16},
\]
with \(\hat\rho(u)=\tfrac{3}{4}(1-u^2)\) the normalized weight. Multiplying the isotropic contraction and the segment ratio gives
\[
\frac{4}{3}\times \frac{5}{16}=\frac{5}{12}.
\]
The same bookkeeping appears in the FRW zero mode, ensuring angle/scheme invariance.

\section{CHM diamond vs.\ half-space KMS deviation}
\label{app:chm-kms-estimate}
In Riemann-normal coordinates about the diamond center,
\[
g_{ab}(x)=\eta_{ab}-\tfrac{1}{3}R_{acbd}(0)\,x^c x^d+\mathcal O\!\big((x/L_{\rm curv})^3\big),
\]
and the CHM conformal-Killing field \(\xi^a_{\rm CHM}\) differs from the exact boost \(\xi^a_{\rm BW}\) by
\[
\delta\xi^a=\mathcal O\!\left(\frac{\ell^2}{L_{\rm curv}^2}\right).
\]
The KMS susceptibility’s \emph{fractional} deviation then scales as
\[
\frac{\delta\chi}{\chi_{\rm BW}}=\mathcal O\!\left(\frac{\ell^2}{L_{\rm curv}^2}\right).
\]
Numerically, \(\ell=10\,\mathrm{pc}\) and \(L_{\rm curv}=10\,\mathrm{Mpc}\) give \((\ell/L_{\rm curv})^2\sim 10^{-10}\), negligible relative to our conservative \(\sim 5\%\) \(\beta\) uncertainty. \emph{A rigorous bound would require a full Hadamard parametrix construction in curved spacetime, beyond our current scope.}

\section{Historical note on Clausius framing (superseded)}
\label{app:clausius-historical}
Earlier drafts expressed the working-order statement in Clausius terms. In this version, macroscopic heat language is removed; normalization is entirely via KMS/FDT in the MI/moment-kill channel.

% ===============================
\bibliographystyle{unsrt}
\begin{thebibliography}{99}

\bibitem{BisognanoWichmann1975}
J.~J.~Bisognano and E.~H.~Wichmann,
``On the Duality Condition for a Hermitian Scalar Field,''
\emph{J. Math. Phys.} \textbf{16}, 985 (1975);
``On the Duality Condition for Quantum Fields,''
\emph{J. Math. Phys.} \textbf{17}, 303 (1976).

\bibitem{Casini2011}
H.~Casini, M.~Huerta, and R.~C.~Myers,
``Towards a derivation of holographic entanglement entropy,''
\emph{JHEP} \textbf{05}, 036 (2011).

\bibitem{OsbornPetkou1994}
H.~Osborn and A.~C.~Petkou,
``Implications of Conformal Invariance in Field Theories for General Dimensions,''
\emph{Annals Phys.} \textbf{231}, 311--362 (1994).

\bibitem{BelliniSawicki2014}
E.~Bellini and I.~Sawicki,
``Maximal freedom at minimum cost: linear large-scale structure in general modifications of gravity,''
\emph{JCAP} \textbf{07}, 050 (2014).

\bibitem{Planck2018}
Planck Collaboration,
``Planck 2018 results. VI. Cosmological parameters,''
\emph{Astron. Astrophys.} \textbf{641}, A6 (2020).

\bibitem{LombriserTaylor2016}
L.~Lombriser and A.~Taylor,
``Breaking a Dark Degeneracy with Gravitational Waves,''
\emph{JCAP} \textbf{03}, 031 (2016).

\end{thebibliography}

\end{document}