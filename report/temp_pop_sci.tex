\documentclass[aps,prd,onecolumn,superscriptaddress,nofootinbib]{revtex4-2}

% ---- Minimal packages
\usepackage[utf8]{inputenc}
\usepackage[T1]{fontenc}
\usepackage{lmodern}
\usepackage{amsmath,amssymb,bm,mathtools}
\usepackage{microtype}
\usepackage[unicode, pdfencoding=auto, psdextra]{hyperref}

% ---- Handy macros
\newcommand{\OmL}{\Omega_\Lambda}
\newcommand{\eps}{\varepsilon}
\newcommand{\Sig}{\Sigma}
\newcommand{\cgeo}{c_{\rm geo}}

\begin{document}

\title{Gravity as Capacity Throttling:\\
A Scientist–Literate Primer (Precursor to Referee Review)}

\author{[clg]}
\affiliation{[Institution(s)]}
\date{\today}

\begin{abstract}
\textbf{Idea in one line:} Gravity responds not only to mass–energy but also to the \emph{finite capacity} of spacetime to store quantum information. As the universe expands, the \emph{capacity load} increases monotonically (an entropic statement), which throttles the effective strength of gravity in a predictable, testable way.

\smallskip
\textbf{Three pillars.}
(1) \emph{Capacity limit adjusts gravity}: tighter capacity $\Rightarrow$ weaker effective $G$; looser capacity $\Rightarrow$ stronger effective $G$.
(2) \emph{Field strength adjusts capacity limit}: strong fields/curvature push the system closer to its limit (throttling back to GR in high-curvature environments); weak fields sit farther from the limit (room for emergent effects).
(3) \emph{Entropy increases monotonically}: a coarse-grained, positive–direction evolution ensures the capacity load never decreases with cosmic time, fixing the sign of corrections and supplying an arrow of time.

\smallskip
\textbf{Why this matters.} The same three principles naturally: (i) reduce late-time growth ($S_8$ band), (ii) soften Hubble-ladder tensions by a small, controlled amount, and (iii) (\emph{optional, exploratory}) explain lensing peak shifts in shocked cluster gas without touching FRW distances. All with \emph{no hand-tuned parameters}.
\end{abstract}

\maketitle

\section*{1. What ``capacity throttling'' means}
Think of a finite-bandwidth channel. When it is lightly loaded, the channel behaves as if it had more headroom; under heavy load it throttles. Our claim is that spacetime has an \emph{information capacity} that plays a similar role. The cosmic state variable $\eps(a)$ (dimensionless; ``capacity load'') \emph{monotonically increases} with the scale factor $a$:
\[
\frac{d\eps}{d\ln a} \ge 0 \qquad \text{(monotonic entropy / capacity load increase).}
\]
This load renormalizes the effective Planck mass and therefore the effective gravitational coupling.

\section*{2. One-line working equation for growth}
At large (sub-horizon, quasi-static) scales we can summarize the modification as a single, testable factor multiplying the Poisson equation:
\begin{align}
\nabla^2\Phi &= 4\pi G a^2 \rho_m \,\mu(\eps,s), \\
\mu(\eps,s) &= \frac{1}{1+\frac{5}{12}\,\eps\,s(x)} \qquad \text{(capacity throttling of gravity)}.
\end{align}
Here $\Phi$ is the Newtonian potential, and $s(x)\!\in[0,1]$ is a \emph{local environment weight} that collapses to $1$ in weak curvature (voids; low Weyl) and to $0$ in strong curvature (Solar System, CMB/BAO regime). Thus:

\begin{itemize}
\item \textbf{Capacity limit adjusts $G$.} The factor $\mu(\eps,s)<1$ weakens effective $G$ when the capacity load $\eps$ is nonzero (throttling).
\item \textbf{Field strength adjusts capacity limit.} In strong fields, $s(x)\!\to\!0$ (no throttling; GR recovered). In weak fields, $s(x)\!\to\!1$ (maximal throttling from the background load).
\item \textbf{Monotonic entropy increase.} Because $d\eps/d\ln a\!\ge\!0$, the correction has a fixed sign and grows mildly over time—crucial for stability and predictivity.
\end{itemize}

Distances and wave speeds remain GR-like at this working order:
\[
\nabla^2\frac{\Phi+\Psi}{2}=4\pi G a^2\rho_m, \qquad c_T=1,
\]
so standard distance ladders (CMB, BAO) stay intact. The \emph{observable} lensing change comes indirectly through the altered growth $D(a)$.

\section*{3. A single picture to keep in mind}
\begin{itemize}
\item \emph{Background (cosmic)}: $\eps(a)$ increases monotonically, throttling growth slightly as the universe ages.
\item \emph{Environment (local)}: $s(x)$ turns the throttling off in strong fields (Solar System; early-time CMB/BAO regime) and on in weak fields (voids; late-time LSS).
\item \emph{Net effect}: later formation is a bit \emph{less efficient} than GR would predict $\Rightarrow$ lower $S_8$ without retuning early-time pillars.
\end{itemize}

\section*{4. Three concise outcomes}
\paragraph*{(A) $S_8$ band (growth).}
Because $\mu(\eps,s)\!\le\!1$ and $d\eps/d\ln a\!\ge\!0$, late-time structure grows slightly less than in GR. Under mild assumptions on monotone $\eps(a)$, this yields a \emph{band} for $S_8$ (early-loaded profiles give the upper edge, late-loaded the lower edge). This is a prior-predictive \emph{interval}, not a fit knob.

\paragraph*{(B) Hubble-ladder softening.}
A small, controlled background throttling modestly reduces the ladder inference for $H_0$ relative to pure GR baselines, nudging ladder and early-time inferences closer without spoiling CMB/BAO distances.

\paragraph*{(C) Optional, local lensing suppression in shocked gas.}
\emph{Only if invoked} (exploratory), strong shears in \emph{shocked intracluster gas} reduce the local lensing response by a bounded factor
\[
\Sig(x)\simeq 1-\alpha_{\rm opt}\,\frac{\mathcal{S}_{\rm shock}(x)}{1+\mathcal{S}_{\rm shock}(x)} \in (0,1],
\]
correlating lensing deficits with X-ray/temperature jumps and radio relics. This is a \emph{separate, environmental} effect that does not alter FRW distances and is \emph{independent} of the background capacity throttling above.

\section*{5. What to measure (minimal, falsifiable tests)}
\begin{enumerate}
\item \textbf{Growth} ($f\sigma_8$, lensing–clustering combinations): look for a consistent \emph{downward shift} within a narrow band set by monotone $\eps(a)$.
\item \textbf{Distances} ($d_L^{\rm EM}$ vs.\ $d_L^{\rm GW}$; CMB/BAO): \emph{no} working-order split or distance distortion (consistency with GR).
\item \textbf{Environment switch} (Solar System, strong-field lenses): \emph{no} deviations—$s(x)\to 0$.
\item \textbf{Clusters (optional channel)}: lensing suppression should \emph{track shock diagnostics} (X-ray edges, radio relics) and \emph{fade} as shocks dissipate.
\end{enumerate}

\section*{6. Clean falsifiers (any one is enough)}
\begin{itemize}
\item A statistically significant \emph{increase} of growth relative to GR at late times (violates $\mu\le 1$ and/or $d\eps/d\ln a\ge 0$).
\item A robust GR-scale \emph{distance} anomaly at working order (e.g.\ $d_L^{\rm GW}\!\neq\!d_L^{\rm EM}$ at the $10^{-3}$ level in quiet cosmology).
\item Solar-System or strong-field lens tests showing deviations (would contradict $s(x)\!\to\!0$).
\item For the optional cluster channel: \emph{no spatial correlation} between lensing suppression and independent shock tracers; or a transport-inferred $\alpha_{\rm opt}$ inconsistent with the required suppression.
\end{itemize}

\section*{7. Minimal math sandbox (for readers who want one line more)}
All the action is in a single factor:
\[
\boxed{\quad \mu(\eps,s)=\frac{1}{1+\frac{5}{12}\,\eps\,s(x)} \quad}
\]
with three rules of thumb:
\begin{enumerate}
\item $\eps(a)$ increases monotonically (entropy/capacity load $\nearrow$ with cosmic time).
\item $s(x)\approx 1$ in weak fields (voids/LSS), $s(x)\approx 0$ in strong fields (CMB/BAO, Solar System).
\item Distances stay GR-like at this order; growth is the \emph{leading} place to look.
\end{enumerate}

\section*{8. One-paragraph provenance (why this is principled)}
Behind this primer sits a referee-grade derivation: a projected modular-response theorem in QFT (fixing the universal $5/12$ weak-field factor), a covariant coarse-graining that yields a positive, contact-like response (monotone capacity load), and an action-level environment weight $s(x)$ that enforces Solar-System safety. The exploratory cluster channel is anchored to standard Schwinger–Keldysh/BRSSS hydrodynamics, making the local lensing suppression a function of transport coefficients rather than a free fit. Readers who want the full derivation can open the technical companion.

\section*{9. Plain-language summary (takeaway)}
\emph{Capacity sets gravity, fields set capacity, entropy only goes up.} These three statements—each independently testable—together explain why gravity looks \emph{exactly} like GR where it must, and only gently deviates where the universe is weakest and emptiest, nudging key cosmological tensions in the right direction without tuning.

\bigskip
\noindent\textit{Companion documents:} (i) “Referee version” (full derivations, proofs, and appendices), (ii) this “Scientist-literate primer.” The two are logically consistent; the primer is a map, the referee draft is the proof.

\end{document}