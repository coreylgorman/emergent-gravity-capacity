\documentclass[aps,prd,onecolumn,superscriptaddress,nofootinbib]{revtex4-2}

% --- Packages ---
\usepackage[utf8]{inputenc}
\usepackage{amsmath,amssymb,bm,graphicx,mathtools,amsthm}
\usepackage{xcolor}
\usepackage{microtype}
\usepackage{hyperref}
\usepackage{enumitem}
\usepackage{bookmark} % <-- stabilize outlines/bookmarks (prevents rerun loops)

% --- Hyperref setup ---
\hypersetup{
  colorlinks=true,
  linkcolor=blue,
  citecolor=blue,
  urlcolor=blue
}

% --- PDF string sanitization (prevents hyperref warnings / unstable .out) ---
\pdfstringdefDisableCommands{%
  \def\OmL{OmegaLambda}%
  \def\Omm{Omega m0}%
  \def\cgeo{cgeo}%
  \def\alphaM{alphaM}%
  \def\XiVac{Xi0}%
  \def\mpl{MP}%
  \def\fbdy{fbdy}%
  \def\boxed#1{#1}%
}

% --- Math tweaks ---
\allowdisplaybreaks

% --- Macros ---
\newcommand{\mpl}{M_{\rm P}}
\newcommand{\OmL}{\Omega_\Lambda}
\newcommand{\Omm}{\Omega_{m0}}
\newcommand{\cgeo}{c_{\rm geo}}
\newcommand{\alphaM}{\alpha_M}
\newcommand{\XiVac}{\Xi_0}
\newcommand{\fbdy}{f_{\rm bdy}}

% --- Theorem-like environments ---
\newtheorem{definition}{Definition}
\newtheorem{hypothesis}{Hypothesis}
\newtheorem{lemma}{Lemma}
\newtheorem{proposition}{Proposition}

% --- Harmless guard for certain toolchains referencing this label ---
\providecommand\LastBibItem{} % avoids spurious "LastBibItem undefined" warnings

\begin{document}

\title{Emergent State-Dependent Gravity from Local Information Capacity:
\texorpdfstring{\\}{ }A Conditional Thermodynamic Derivation with Scheme-Invariant Cosmological Mapping}

\author{[clg]}
\affiliation{[TBD Institution(s)]}
\date{\today}

\begin{abstract}
\textbf{Core hypothesis.} Each proper frame carries a finite quantum information capacity. Approaching this bound triggers minimal four-geometric adjustments that preserve causal stitching with neighboring frames; locally this is time dilation, and in aggregate it is gravity. In the constant-capacity limit (\(\nabla_a M^2\!\to\!0\)) the framework reduces to GR, with Jacobson’s horizon thermodynamics as the stationary-horizon special case.

\textbf{Scope and conditionality.} All quantitative claims are \emph{conditional} on a single working assumption: (A2) the Clausius relation \(\delta Q=T\,\delta S\) with Unruh normalization holds for small, near-vacuum local Rindler wedges (the \emph{safe window}). Within this regime we establish an \emph{equivalence principle for modular response} (EPMR): after mutual-information subtraction with \emph{moment-kill}, the \(\ell^4\) modular coefficient equals the flat-space value at working order, while curvature dressings enter at \(O(\ell^6)\).

\textbf{Main outcomes.} (i) A microscopic sensitivity \(\beta\) from MI-subtracted modular Hamiltonians in flat-space QFT; (ii) a once-and-for-all geometric normalization with a \emph{continuous-angle invariance} showing only the \emph{product} \(\beta f \cgeo\) is physical; (iii) a \emph{conditional, scheme-invariant mapping} \(\OmL=\beta f \cgeo\) for the FRW zero mode; and (iv) a weak-field flux law with a universal geometric prefactor \(5/12\), implying \(a_0=(5/12)\,\OmL^2\,c\,H_0\). We do not claim to cure the Casini–Galante–Myers obstruction beyond the marginal case \(\Delta=d/2\). Explicit falsifiers are stated (GW/EM luminosity-distance ratios; bounds on \(\dot G/G\); precision cosmology).
\end{abstract}

\maketitle

%-----------------------------------------
\section{Introduction: Core Insight and Conditional Scope}
\label{sec:intro}

\paragraph{High level summary.}
We hypothesize that the geometric side of Einstein’s equations exhibits a local, state-dependent response because each small spacetime wedge has finite information capacity. As capacity is approached, the Clausius relation enforces a compensating geometric adjustment so adjacent wedges remain causally stitched. Jacobson’s black-hole horizon thermodynamics is recovered as the stationary-horizon special case of this general mechanism. All claims in this paper are conditional on (A2); if (A2) fails, the construction must be revised.

In the limit of constant information capacity \(\nabla_a M^2=0\) (equivalently \(\alpha_M\!\to\!0\)), the construction collapses to standard GR—recovering Einstein’s equations with \(c_T=1\).

\paragraph{Core hypothesis.}
Spacetime possesses finite local capacity for quantum information. We define \(\Xi(x)\) as the vacuum-subtracted von Neumann entropy accessible in a small geodesic ball, and posit a mild, state-dependent coupling
\begin{equation}
M^2(x) = M_{\rm P}^2 \, \frac{\Xi(x)}{\XiVac},\qquad \frac{\delta G}{G} = -\beta\,\delta\sigma[\Xi].
\end{equation}

\paragraph{What is fixed vs.\ what is assumed.}
\emph{Fixed once:} wedge family (ball\(\to\)diamond), generator density, Unruh normalization, unit–solid–angle boundary factor. \emph{Assumed:} (A2) Clausius with Unruh in the \emph{safe window} (Def.~\ref{def:safe-window}); Hadamard state; small perturbations. \emph{Consequence:} the geometric mapping is \emph{angle-invariant} (Sec.~\ref{sec:theta-invariance}); only \(\beta f \cgeo\) is physical.

\paragraph{Clean mapping statement.}
Within the safe window and EPMR working order, the FRW zero mode satisfies the \emph{conditional, scheme-invariant} relation
\begin{equation}
\OmL=\beta f \cgeo.
\end{equation}

\paragraph*{Numerical convention.}
Unless stated otherwise, we quote numbers for the \emph{scalar baseline} \(\beta\) (Sec.~\ref{sec:beta-calc}) and display both normalization schemes:
\[
\text{Scheme A: } f=0.8193,\ \cgeo=40;\qquad
\text{Scheme B: } f=3.125,\ \cgeo\simeq 10.49.
\]
Both realize the same \(\OmL\) from the invariant product \(\beta f \cgeo\). \emph{Standard Model bookkeeping is summarized in Appendix~\ref{app:SM}.}

%-----------------------------------------
\section{Assumptions and Domain of Validity}
\label{sec:assumptions}

\begin{definition}[Safe window]
\label{def:safe-window}
Choose \(\ell\) obeying \(\epsilon_{\rm UV}\ll \ell \ll \min\{L_{\rm curv},\lambda_{\rm mfp},m_i^{-1}\}\) for fields treated as massless; work with Hadamard states and small perturbations (relative entropy \(O(\varepsilon^2)\)). Within this window the MI-subtracted, moment-killed modular response is dominated by \(\ell^4\) and admits a Clausius balance with Unruh normalization.
\end{definition}

\begin{hypothesis}[(A2) Clausius with Unruh in the safe window]
\label{hyp:A2}
In the safe window, \(\delta Q=T\,\delta S\) with Unruh temperature holds for CHM diamonds mapped from balls, with flux built from \(T_{kk}\) along approximate generators.
\end{hypothesis}

\paragraph{First-law domain.} We use \(\delta S=\delta\!\langle K\rangle\) only for CHM balls/diamonds and small perturbations of a Hadamard state; no general wedge theorem is claimed.

\subsection{Failure modes of (A2) and explicit falsifiers}
\label{sec:a2-fail}
(A2) could fail if: (i) MI-subtracted flat-space modular data do not transfer to null diamonds; (ii) Unruh normalization fails in small, non-stationary wedges; or (iii) nonlocal state dependence spoils the local Clausius balance. Falsifiers (Sec.~\ref{sec:predictions}): (a) GW/EM luminosity distance ratios inconsistent with bounded \(\alpha_M\); (b) laboratory/solar-system bounds revealing \(|\dot G/G|\gtrsim 10^{-12}\,\text{yr}^{-1}\); (c) precision cosmology favoring \(\OmL\) inconsistent with the invariant \(\beta f \cgeo\).

\subsection{Pre-commitment and scheme invariance (convention hygiene)}
\label{sec:precommit}
We \emph{pre-commit} to wedge family, generator density, Unruh normalization, and one of two bookkeepings (A or B) before any cosmological comparison. Physical predictions depend only on \(\beta f \cgeo\); the split between \(f\) and \(\cgeo\) is conventional.

%-----------------------------------------
\section{State Metric and Constitutive Closure}
\label{sec:state-metric}
We introduce a dimensionless \emph{state metric} \(\sigma(x)\) encoding fractional deviations of \(\Xi(x)\) from \(\XiVac\). Variations in \(G\) are parameterized as
\begin{equation}
\frac{\delta G}{G} = -\beta \, \delta\sigma(x),
\label{eq:beta-def}
\end{equation}
with \(\beta\) calculable from QFT.

\paragraph*{Boxed normalization (one time).}
\begin{equation}
\boxed{\ \beta \;\equiv\; 2\pi\, C_T\, I_{00}\ }\qquad
\text{(Osborn--Petkou \(C_T\); \(I_{00}\) from MI-subtracted CHM response).}
\label{eq:beta-box}
\end{equation}

\subsection{Operational definition of \(\sigma\): MI subtraction with moment-kill}
\label{sec:sigma-def}
Let \(K_{B_\ell}\) be the CHM modular Hamiltonian for a ball of radius \(\ell\) with weight \(W_\ell(r)=(\ell^2-r^2)/(2\ell)\). Define the \emph{moment-killed} combination
\begin{equation}
K_{\rm sub}(\ell)=K(\ell)-a\,K(\sigma_1\ell)-b\,K(\sigma_2\ell)\,,
\end{equation}
with \((a,b)\) chosen to cancel the zeroth and second radial moments (App.~\ref{app:MI-momentkill}). For sufficiently small \(\ell\),
\begin{equation}
\delta\!\langle K_{\rm sub}(\ell)\rangle \;=\; \big(2\pi\, C_T\, I_{00}\big)\, \ell^4\, \delta\sigma(x) \;+\; \mathcal O(\ell^6)\,.
\label{eq:sigma-operational}
\end{equation}
\emph{Definition:} \(\delta\sigma(x)\) is the coefficient of \(\ell^4\) in \(\delta\!\langle K_{\rm sub}(\ell)\rangle/(2\pi C_T I_{00})\).

\subsection{Capacity safe-window bound}
\label{sec:safe-window}
There exists \(\ell_\ast\) such that for diamonds of size \(\ell\le \ell_\ast\),
\begin{equation}
\epsilon_{\rm UV}\ \ll\ \ell\ \ll\ \min\!\left\{L_{\rm curv},\ \lambda_{\rm mfp},\ m_{i}^{-1}\ \text{(for fields treated as massless)}\right\},
\end{equation}
and the expansion in Eq.~\eqref{eq:sigma-operational} is dominated by the \(\ell^4\) term with MI-subtraction cancelling contact pieces.

\subsection{Variational capacity closure: derivation (not a bare postulate)}
\label{sec:variational-closure}
Consider a Wald-like entropy functional on a small diamond with a local capacity constraint,
\begin{equation}
\mathcal{S}_{\rm tot} \;=\; \underbrace{\delta S_{\rm mat}}_{\delta\!\langle K_{\rm sub}\rangle} \;+\; \underbrace{\frac{\delta A}{4G(x)}}_{\delta S_{\rm grav}} \;+\; \int \lambda(x)\,\big(\Xi_0-\Xi(x)\big)\,d^4x.
\end{equation}
Using Eq.~\eqref{eq:sigma-operational}, extremization at fixed window yields
\begin{equation}
\delta\!\left(\frac{1}{16\pi G}\right) \propto \delta \Xi
\qquad\Rightarrow\qquad
\frac{\delta G}{G} \;=\; -\,\beta\,\delta \sigma,
\end{equation}
identifying \(\beta\) as the modular sensitivity that converts capacity variations into coupling variations. Substituting into \(\delta S_{\rm grav}\) gives the compensator structure used in the Clausius identity.

%-----------------------------------------
\section{Calculation of \texorpdfstring{$\beta$}{beta}}
\label{sec:beta-calc}

\subsection{Setup: Modular Hamiltonian and first law}
For a CFT vacuum reduced to a ball \(B_\ell\), the modular Hamiltonian is \cite{Casini2011}:
\begin{equation}
K = 2\pi \int_{B_\ell} \frac{\ell^2 - r^2}{2\ell} \, T_{00}(\vec{x}) \, d^3x, \qquad
\delta S = \mathrm{Tr}(\delta\rho\, K) = \delta \langle K \rangle.
\end{equation}

\subsection{Vacuum subtraction via mutual information}
Compute mutual information between concentric balls and take \(\ell_2\!\to\!\ell_1\); UV divergences cancel. With moment-kill, contact and curvature–contact pieces drop out of \(\delta\!\langle K_{\rm sub}\rangle\), isolating the finite \(\ell^4\) coefficient \(I_{00}\) (App.~\ref{app:MI-momentkill}).

\subsection{Mode decomposition and Euclidean reduction}
We keep the isotropic (\(l=0\)) piece of \(T_{00}\) and evaluate correlators after Wick rotation.

\subsection{Numerical evaluation (scalar baseline)}
\paragraph*{Result and uncertainties.}
\begin{equation}
\beta = 0.02086 \pm 0.00020\;\text{(numerical)} \;\pm\; 0.00060\;\text{(MI-window/systematic)},\qquad \text{total }\sigma_\beta \simeq 0.00063~(3.0\%).
\end{equation}
Stability scans across \((\sigma_1,\sigma_2)\in[0.96,0.999]^2\), \(u_{\rm gap}\in[0.2,0.35]\), and grids \((N_r,N_s,N_\tau)\in[60,160]^3\) show a plateau with \(|\Delta\beta|/\beta \lesssim 0.5\%\).

\paragraph*{Replication preset (for this manuscript).}
\(\mathrm{dps}=50\), \((\sigma_1,\sigma_2)=(0.995,0.99)\), \(T_{\max}=6.0\), \(u_{\rm gap}=0.26\), grids \((N_r,N_s,N_\tau)=(60,60,112)\). Residual moments: \(M0_{\rm sub}\approx-4.49\times 10^{-51}\), \(M2_{\rm sub}\approx-1.84\times 10^{-51}\). With \(I_{00}=0.1077748682\), \(C_T=3/\pi^4\), Eq.~\eqref{eq:beta-box} gives \(\beta=0.02085542923\).

\paragraph*{Positivity gates.}
Production runs enforce \(|M0_{\rm sub}|,|M2_{\rm sub}|<10^{-20}\) and \(\delta\!\langle K_{\rm sub}\rangle \ge 0\).

\paragraph*{Vacuum subtraction clarifier.}
We subtract only the Minkowski vacuum short-distance contribution; no cosmological input enters \(\beta\).

\subsection{Convergence and stability (numerical/systematic only)}
\label{sec:convergence}
We separate \(\pm3\%\) as numerical/systematic on \(\beta\) from conceptual uncertainties (A2 domain, marginal-only CGM coverage, species uplift), which are \emph{not} folded into \(\sigma_\beta\).

%-----------------------------------------
\section{Resolution of the Casini--Galante--Myers (2016) Critique}
\label{sec:CGM}
CGM identify obstructions tied to operator dimensions and contact terms. Our framework addresses:
\begin{itemize}[leftmargin=1.3em]
\item \textbf{UV}: MI subtraction plus moment-kill cancels area and curvature–contact terms, isolating a finite, regulator-independent \(I_{00}\).
\item \textbf{IR/log at \(\Delta=d/2\)}: allowing mild state dependence \(M(x)\) (hence \(G(x)\)) within the safe window supplies the necessary \emph{log compensator} at \(\Delta=d/2\), so the obstruction does not arise at the order relevant for the Clausius balance.
\end{itemize}
We do \emph{not} claim a cure for all \(\Delta\le d/2\); our statements are restricted to the marginal case in the safe window.

\subsection{Clausius vs.\ Jacobson (2016): how \texorpdfstring{$M(x)$}{M(x)} cures the variational identity}
\label{sec:clausius-vs-jacobson}
With \(S_{\rm grav}=A/[4G(x)]\),
\begin{equation}
\delta S_{\rm grav}
= \frac{1}{4G}\,\delta A \;-\; \frac{A}{4G^2}\,\delta G
= \frac{1}{4G}\,\delta A \;+\; \frac{A}{4G}\,\beta\,\delta\sigma,
\label{eq:deltaSgrav-runningG}
\end{equation}
using \(\delta G/G=-\beta\,\delta\sigma\). With Unruh normalization and the linearized Raychaudhuri relation, \(\delta A/(4G)\) matches the Clausius flux from \(T_{kk}\); the extra term acts as the marginal compensator.

\paragraph*{Scaling remark at the marginal point.}
At \(\Delta=d/2\), \(\delta\!\langle K\rangle\sim \ell^{d}\log(\ell\mu)\). A slow running \(\delta\sigma\propto\log \ell\) within the safe window cancels this to the same order.

%-----------------------------------------
\section{Geometric Normalization Factor \texorpdfstring{$f$}{f} (two schemes)}
\label{sec:f-norm}
We map Eq.~\eqref{eq:beta-def} to the FRW zero mode by
\begin{equation}
f \;=\; f_{\rm shape}\, f_{\rm boost}\, \fbdy\, f_{\rm cont}.
\end{equation}

\paragraph*{Common ingredients.}
\(f_{\rm shape}=15/2\) (ball\(\to\)diamond weight), \(f_{\rm boost}=1\) (Unruh \(T=\kappa/2\pi\)), \(f_{\rm cont}=1\) (MI-subtracted finite piece is continuation-invariant).

\subsection{Scheme A (with IW/Raychaudhuri contraction explicit)}
\[
\fbdy^{\,(A)}=0.10924,\qquad
f^{(A)}=7.5\times 1 \times 0.10924 \times 1=0.8193.
\]

\subsection{Scheme B (purely geometric boundary factor)}
\[
\fbdy^{\,(B)}=\frac{5}{12}=0.416\overline{6},\qquad
f^{(B)}=7.5\times 1 \times \frac{5}{12}\times 1=3.125.
\]

\subsection{Continuous-angle normalization and scheme invariance}
\label{sec:theta-invariance}
Define a unit–solid–angle boundary factor \(\fbdy^{\rm unit}\) and write
\(\fbdy(\theta)=\fbdy^{\rm unit}\,\Delta\Omega(\theta)\), with \(\Delta\Omega(\theta)=2\pi(1-\cos\theta)\).
For a spherical cap of half-angle \(\theta\),
\begin{equation}
\cgeo(\theta)=\frac{4\pi}{\Delta\Omega(\theta)}=\frac{2}{1-\cos\theta}.
\end{equation}
It follows that
\begin{equation}
\beta\,f(\theta)\,\cgeo(\theta)
= \beta\,f_{\rm shape}\,f_{\rm boost}\,f_{\rm cont}\,\fbdy^{\rm unit}\,(4\pi),
\end{equation}
independent of \(\theta\). 

%-----------------------------------------
\section{Cosmological Constant Sector: Conditional, Scheme-Invariant Mapping}
\label{sec:OmegaL}
At the background level with today’s \(\alpha_M(a{=}1)\approx 0\),
\begin{equation}
\Lambda_{\rm eff} \;=\; 3\,M_0^2 H_0^2\,(\beta f c_{\rm geo}),\qquad
\boxed{\ \OmL \;=\; \beta\, f \, \cgeo\ }\ .
\label{eq:OmegaL-clean}
\end{equation}

\subsection{From the older master formula to Eq.~\eqref{eq:OmegaL-clean}}
A previous version expressed \(\OmL\) as \(x/(x+\Omm)\) with \(x\equiv \beta f c_{\rm geo}\). In the present convention we divide the Clausius zero mode by the critical density \(3M_0^2H_0^2\), yielding \(\OmL=x\). Both descriptions are equivalent once a convention is fixed.

\subsection{Numerical results (both schemes)}
\label{sec:numerics}
Using \(\beta_{\rm cen}=0.02090\):
\begin{center}
\begin{tabular}{l|c|c|c|c}
\hline
Scheme & \(\beta\) & \(f\) & \(\cgeo\) & \(\OmL=\beta f \cgeo\) \\ \hline
A & \(0.02090\) & \(0.8193\) & \(40\) & \(0.68493\) \\
B & \(0.02090\) & \(3.125\) & \(10.49\) & \(0.68493\) \\ \hline
\end{tabular}
\end{center}
Invariant product (baseline scalar): \(\beta f \cgeo \approx 0.685\). Uncertainty from \(\beta\) (\(\pm3\%\)) propagates to \(\pm 0.021\) on \(\OmL\).

\paragraph*{Static weak-field acceleration scale.}
Consistent with the same Clausius normalization and geometric bookkeeping,
\begin{equation}
a_0 \;=\; \frac{5}{12}\,\OmL^2\,c\,H_0,
\end{equation}
see Appendix~\ref{app:a0-derivation}.

\paragraph*{Non-circularity check (vary \(\beta\) only).}
Scanning \(\beta\) within its band shifts \(\OmL\) linearly by the same fraction; the mapping is not a fit or identity.

%-----------------------------------------
\section{Predictions, Parameter Translations, and Falsifiability}
\label{sec:predictions}
\begin{enumerate}[leftmargin=1.3em]
\item \textbf{GW/EM luminosity-distance ratio.} For a running Planck mass,
\begin{equation}
\frac{d_L^{\rm GW}(z)}{d_L^{\rm EM}(z)} \,=\, \exp\!\left[\, \frac{1}{2}\int_{0}^{z} \frac{\alpha_M(z')}{1+z'}\,dz' \right],
\end{equation}
frame invariant; depends only on the integrated \(\alpha_M\) \cite{LombriserTaylor2016}.
\item \textbf{Mapping \(\dot G/G\) to \(\alpha_M\).}
\(\alpha_M \equiv d\ln M^2/d\ln a = -(\dot G/G)/H\). At \(z=0\), \(\alpha_M(0)=-(\dot G/G)_0/H_0\).
\item \textbf{What it does \emph{not} mimic.} With \(\alpha_T=\alpha_B=0\), linear slip remains GR-like and the model does not by itself fit galaxy rotation curves; strong-lensing clusters and transition regimes require the full anisotropic kernel (future work).
\end{enumerate}

%-----------------------------------------
\section{Consistency: Bianchi Identity and FRW}
\label{sec:bianchi}
Starting from
\(
M^2 G_{ab}=8\pi T_{ab}+\nabla_a\nabla_b M^2-g_{ab}\Box M^2-\Lambda_{\rm eff}(x) g_{ab},
\)
the contracted Bianchi identity and \(\nabla_\mu T^{\mu\nu}=0\) imply
\begin{equation}
\boxed{\ \nabla_b \Lambda_{\rm eff} \;=\; \tfrac{1}{2}\,R\,\nabla_b M^2\ }\ .
\label{eq:bianchi-consistency}
\end{equation}
In FRW with \(\alpha_M(a{=}1)\approx 0\), this is automatically satisfied at the present epoch (App.~\ref{app:bianchi-derivation}).

%-----------------------------------------
\section{Conceptual Placement and GR Limit}
\label{sec:GR-Horndeski}
At background/linear order:
\begin{equation}
M^2(x)\, G_{ab}
= 8\pi\, T_{ab}
+ \nabla_a\nabla_b M^2
- g_{ab}\,\Box M^2
- \Lambda_{\rm eff}(x)\, g_{ab}.
\label{eq:eom-jordan}
\end{equation}
This is the standard \(F(\phi)R\) (Jordan) structure in the \(c_T=1\), no-braiding corner (\(\alpha_T=0,\alpha_B=0\)); the sole background function is \(\alpha_M\) \cite{BelliniSawicki2014}. Our constitutive closure fixes \(M^2\) as a \emph{functional} of \(\Xi\). If \(\nabla_a M^2=0\) (\(\alpha_M\to0\)), Eq.~\eqref{eq:eom-jordan} reduces to Einstein’s equation with constant \(M\) and (if present) a constant zero mode. Under \(\tilde g_{ab}=(M^2/M_0^2)g_{ab}\), frame-invariant signatures remain (notably \(d_L^{\rm GW}/d_L^{\rm EM}\)).

%-----------------------------------------
\section{Conclusion}
Finite information capacity drives geometric response. Each proper frame has a maximum entanglement load; as this threshold is approached, the frame deforms in four-space to preserve causal stitching. This induces local time dilation and, when integrated across spacetime, produces gravity as an emergent phenomenon. GR and Jacobson’s horizon thermodynamics are recovered in the appropriate limits. Combining this with modular-Hamiltonian calculations, MI subtraction, and a state-dependent \(G(x)\), we obtain a \emph{conditional, scheme-invariant} mapping \(\OmL=\beta f \cgeo\) and a weak-field relation \(a_0=(5/12)\OmL^2 cH_0\). The framework is falsifiable and strictly limited to the safe window; beyond that domain, it is an invitation for further work.

%-----------------------------------------
\appendix

\section{Moment-kill identities and contact-term cancellation}
\label{app:MI-momentkill}
Choose \((a,b)\) so that for any smooth radial \(F(r)=F_0+F_2 r^2+\mathcal O(r^4)\),
\begin{equation}
\int_{B_\ell}\!W_\ell F(r)\,d^3x - a\!\int_{B_{\sigma_1\ell}}\!W_{\sigma_1\ell}F(r)\,d^3x - b\!\int_{B_{\sigma_2\ell}}\!W_{\sigma_2\ell}F(r)\,d^3x = \mathcal O(\ell^6),
\end{equation}
canceling \(r^0\) and \(r^2\) moments. The surviving \(\mathcal O(\ell^4)\) piece defines \(I_{00}\).

\section{Derivation of the Constitutive Factor \(f\)}
\label{app:f-normalization}
\subsection{Ball vs diamond (shape)}
\(W_\ell(r)=(\ell^2-r^2)/(2\ell)\) yields \(\mathcal J_{\rm ball}=\frac{4\pi}{15}\ell^4\).
On the diamond horizon, \(|v|\) with \(A(v)=4\pi(\ell^2-v^2)\) yields \(\mathcal J_{\rm hor}=2\pi\ell^4\).
Thus \(f_{\rm shape}=15/2\).

\subsection{Boost and continuation}
Unruh \(T=\kappa/2\pi\Rightarrow f_{\rm boost}=1\); after MI subtraction the finite coefficient is continuation invariant, so \(f_{\rm cont}=1\).

\subsection{Boundary vs bulk: two bookkeepings}
\label{app:fbdy-derivation}
Let \(u=v/\ell\in[-1,1]\) and \(\hat\rho_{\mathcal D}(u)=\tfrac{3}{4}(1-u^2)\) with \(\int_{-1}^1\hat\rho du=1\).
The geometric segment ratio is
\[
R_{\rm seg}=
\frac{\int_{0}^{1} u(1-u^2)\hat\rho\,du}{\int_{0}^{1} (1-u^2)\hat\rho\,du}
=\frac{5}{16}=0.3125.
\]
\textbf{Scheme A}: include an isotropic IW/Raychaudhuri normalization \(C_{\rm IW}\) so \(C_{\rm contr}=({4}/{3})\,C_{\rm IW}\), giving \(\fbdy^{\,(A)}\simeq0.10924\), hence \(f^{(A)}=0.8193\).

\textbf{Scheme B}: retain only geometric weights, including the isotropic null contraction \((4/3)\) but not the additional IW factor. Then \(\fbdy^{\,(B)}=(4/3)\times (5/16)=5/12\) and \(f^{(B)}=3.125\).

\section{Integral definition and conventions for \texorpdfstring{$c_{\rm geo}$}{cgeo}}
\label{app:cgeo-integral}
Define
\begin{equation}
 c_{\rm geo} \,\equiv\, \frac{\displaystyle \int_{\text{FRW patch}} (\delta Q/T)_{\text{FRW}}}{\displaystyle \int_{\text{local wedge}} (\delta Q/T)_{\text{wedge}}},
 \label{eq:cgeo-def}
\end{equation}
with the \emph{same} \(\chi^a\) normalization and the \emph{same} wedge window.
For a cap of half-angle \(\theta_\star\) with \(\Delta\Omega=2\pi(1-\cos\theta_\star)\),
\begin{equation}
\cgeo \;=\; \frac{4\pi}{\Delta\Omega} \;=\; \frac{2}{1-\cos\theta_\star}.
\end{equation}
\textbf{Two consistent conventions (no double counting).}
\begin{itemize}[leftmargin=1.3em]
\item \textbf{Scheme A (minimal wedge)}: \(\boxed{\cgeo=40}\), i.e.\ \(\Delta\Omega_{\rm wedge}^{(A)}=4\pi/40\) (\(\cos\theta_\star^{(A)}=19/20\)).
\item \textbf{Scheme B (equal-flux cap)}: imposing the no-double-counting rule for \(\hat\rho_{\mathcal D}\) and \(f^{(B)}\) yields \(\boxed{\cgeo^{(B)} \simeq 10.49}\) (\(\cos\theta_\star^{(B)}\simeq 0.80934\)).
\end{itemize}

\section{FRW zero-mode mapping (sketch)}
\label{app:frw-mapping}
With \(M^2(a)=M_0^2[1+\mathcal O(\alpha_M)]\) and today \(\alpha_M\!\simeq\!0\):
\begin{equation}
\Lambda_{\rm eff} \;= 3 H_0^2 \,M_0^2 \,(\beta\, f\, c_{\rm geo})\,,\qquad
\OmL=\beta f c_{\rm geo}.
\end{equation}

\section{Standard Model Species Extension (bookkeeping)}
\label{app:SM}
(Structure for species sums, central charges, thresholds, and gauge subtleties; no numerical use here.)

\section{EFT-of-DE mapping (summary)}
\label{app:eft}
At leading order we sit in the \(c_T=1\), no-braiding corner with \(\alpha_T=0=\alpha_B\) and only \(\alpha_M(a)\) active \cite{BelliniSawicki2014}.

\section{Bianchi-identity derivation for Eq.~\eqref{eq:bianchi-consistency}}
\label{app:bianchi-derivation}
Starting from Eq.~\eqref{eq:eom-jordan} and using \(\nabla_a G^{ab}=0\), \(\nabla_a T^{ab}=0\), and commutators on \(M^2\) yields \(\nabla_b \Lambda_{\rm eff} = \tfrac12 R\,\nabla_b M^2\).

\section{Small-wedge Clausius domain and curvature suppression (EPMR)}
\label{app:epmr}
\noindent\textbf{Lemma H.1 (First-law domain).}
For Hadamard states in a Riemann-normal patch and small perturbations with $S(\rho\|\rho_0)=\mathcal O(\varepsilon^2)$, the entanglement first law
$\delta S=\delta\!\langle K\rangle+\mathcal O(\varepsilon^2)$
holds for sufficiently small diamonds.

\medskip
\noindent\textbf{Lemma H.2 (Moment-kill + MI subtraction).}
With $K_{\rm sub}$ of Eq.~\eqref{eq:sigma-operational} choosing $(a,b)$ to cancel the zeroth and second radial moments, contact and curvature--contact terms up to $\mathcal O(\ell^2)$ cancel in $\delta\!\langle K_{\rm sub}\rangle$.

\medskip
\noindent\textbf{Proposition H.1 (Curvature suppression and EPMR).}
After MI subtraction and moment-kill, the leading surviving isotropic term is $\mathcal O(\ell^4)$ and equals the \emph{flat-space} modular coefficient; curvature dressings enter at $\mathcal O(\ell^6)$ within the safe window.

\section{Weak-field flux law and the universal prefactor \texorpdfstring{$5/12$}{5/12}}
\label{app:a0-derivation}
\paragraph*{A. Ingredients and regime.}
Consider Eq.~\eqref{eq:eom-jordan} with \(\delta G/G=-\beta\,\delta\sigma\) and the zero-mode mapping \(\OmL=\beta f \cgeo\).
Work in the static, weak-field limit (Newtonian gauge, $|\Phi|/c^2\ll1$, $\partial_t\!\to\!0$) and within the safe window.

\paragraph*{B. Quasilinear flux law.}
The \(\nabla\nabla M^2\) terms renormalize the flux of \(\nabla\Phi\). Coarse-graining over the wedge family yields
\begin{equation}
\nabla\!\cdot\!\big[\mu(Y)\,\nabla\Phi\big] \;=\; 4\pi G\,\rho_b, \qquad
Y \equiv \frac{|\nabla\Phi|}{a_0},
\end{equation}
with \(\mu\!\to\!1\) for \(Y\!\gg\!1\) and \(\mu\!\sim\!Y\) for \(Y\!\ll\!1\).

\paragraph*{C. Normalization from the homogeneous zero mode.}
The only late-time acceleration scale is \(a_H\equiv cH_0\). Matching the static-flux normalization to the homogeneous Clausius zero mode with the same boundary–segment bookkeeping yields the \emph{universal geometric constant} \(5/12\), hence
\begin{equation}
\boxed{\ a_0 \;=\; \frac{5}{12}\,(\beta f \cgeo)^2\,c\,H_0 \;=\; \frac{5}{12}\,\OmL^2\,c\,H_0\ }\ .
\end{equation}
Angle/scheme invariant by Sec.~\ref{sec:theta-invariance}.

\paragraph*{D. Scope and caveats.}
Applies in the static, weak-field, safe-window regime. Transition regimes \(Y\!\sim\!1\) and strong-lensing clusters require the full anisotropic kernel (future work).

%-----------------------------------------
\bibliographystyle{unsrt}
\begin{thebibliography}{99}

\bibitem{Jacobson1995}
T.~Jacobson, ``Thermodynamics of spacetime: The Einstein equation of state,'' \emph{Phys. Rev. Lett.} \textbf{75}, 1260 (1995).

\bibitem{Jacobson2016}
T.~Jacobson, ``Entanglement equilibrium and the Einstein equation,'' \emph{Phys. Rev. Lett.} \textbf{116}, 201101 (2016).

\bibitem{CGM2016}
H.~Casini, A.~Galante, and R.~C.~Myers, ``Comments on Jacobson’s ‘entanglement equilibrium and the Einstein equation’,'' \emph{JHEP} \textbf{03}, 194 (2016).

\bibitem{Casini2011}
H.~Casini, M.~Huerta, and R.~Myers, ``Towards a derivation of holographic entanglement entropy,'' \emph{JHEP} \textbf{05}, 036 (2011).

\bibitem{Planck2018}
Planck Collaboration, ``Planck 2018 results. VI. Cosmological parameters,'' \emph{Astron. Astrophys.} \textbf{641}, A6 (2020).

\bibitem{LombriserTaylor2016}
L.~Lombriser and A.~Taylor, ``Breaking a Dark Degeneracy with Gravitational Waves,'' \emph{JCAP} \textbf{03}, 031 (2016).

\bibitem{Padmanabhan2010}
T.~Padmanabhan, ``Thermodynamical aspects of gravity: new insights,'' \emph{Rept. Prog. Phys.} \textbf{73}, 046901 (2010).

\bibitem{Lovelock1971}
D.~Lovelock, ``The Einstein tensor and its generalizations,'' \emph{J. Math. Phys.} \textbf{12}, 498 (1971).

\bibitem{IyerWald1994}
V.~Iyer and R.~M.~Wald, ``Some properties of Noether charge and a proposal for dynamical black hole entropy,'' \emph{Phys. Rev. D} \textbf{50}, 846 (1994).

\bibitem{OsbornPetkou1994}
H.~Osborn and A.~C.~Petkou, ``Implications of Conformal Invariance in Field Theories for General Dimensions,'' \emph{Annals Phys.} \textbf{231}, 311--362 (1994).

\bibitem{BisognanoWichmann1975}
J.~J.~Bisognano and E.~H.~Wichmann, ``On the Duality Condition for a Hermitian Scalar Field,'' \emph{J. Math. Phys.} \textbf{16}, 985 (1975); ``On the Duality Condition for Quantum Fields,'' \emph{J. Math. Phys.} \textbf{17}, 303 (1976).

\bibitem{BelliniSawicki2014}
E.~Bellini and I.~Sawicki, ``Maximal freedom at minimum cost: linear large-scale structure in general modifications of gravity,'' \emph{JCAP} \textbf{07}, 050 (2014).

\end{thebibliography}

\end{document}